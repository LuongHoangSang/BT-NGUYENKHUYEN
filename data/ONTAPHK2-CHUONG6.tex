\begin{center}
	\textbf{ÔN TẬP KIỂM TRA HỌC KÌ 2 NĂM HỌC 2024 -2025}\\
	\textbf{CHƯƠNG 6: NĂNG LƯỢNG}
\end{center}
\Opensolutionfile{ans}[ans/ONTAPHK2-TN]

% ===================================================================
\begin{ex}
	Đại lượng đặc trưng cho tốc độ sinh công của lực là
	\choice
	{\True công suất}
	{công}
	{lực kéo}
	{hiệu suất}
	\loigiai{}
\end{ex}
% ===================================================================
\begin{ex}
	Một động cơ điện được thiết kế để kéo một thùng than khối lượng \SI{400}{\kilogram} từ dưới mỏ có độ sâu \SI{1200}{\meter} lên mặt đất trong thời gian 2 phút. Hiệu suất của động cơ là \SI{80}{\percent}. Lấy $g=\SI{9.8}{\meter/\second^2}$. Công suất toàn phần của động cơ là
	\choice
	{\SI{7.8}{\kilo\watt}}
	{\SI{9.8}{\kilo\watt}}
	{\SI{31}{\kilo\watt}}
	{\True \SI{49}{\kilo\watt}}
	\loigiai{}
\end{ex}
% ===================================================================
\begin{ex}
	Trong công trường xây dựng, một máy nâng có công suất \SI{1.2}{\kilo\watt} đang làm việc, khi máy nâng đều một vật khối lượng \SI{200}{\kilo\gram} lên độ cao \SI{18}{\meter} thì mất thời gian là \SI{90}{\second}. Lấy $g=\SI{10}{\meter/\second^2}$. Hiệu suất của máy nâng xấp xỉ là
	\choice
	{\True \SI{33.33}{\percent}}
	{\SI{45}{\percent}}
	{\SI{50}{\percent}}
	{\SI{66.67}{\percent}}
	\loigiai{}
\end{ex}



% ===================================================================
\begin{ex}
	Một vật có khối lượng \SI{60}{\kilogram} đang chuyển động với tốc độ \SI{2.5}{\meter/\second}. Động năng của vật là
	\choice
	{\SI{60}{\joule}}
	{\SI{552.5}{\joule}}
	{\SI{150}{\joule}}
	{\True \SI{187.5}{\joule}}
	\loigiai{}
\end{ex}

% ===================================================================
\begin{ex}
	Từ độ cao \SI{2.0}{\meter} so với mặt đất, người ta ném một vật khối lượng \SI{100}{\gram} thẳng đứng lên cao với vận tốc đầu là \SI{3}{\meter/\second}. Bỏ qua lực cản của không khí. Lấy $g=\SI{10}{\meter/\second^2}$. Xác định cơ năng của vật tại vị trí cao nhất mà vật đạt tới
	\choice
	{\SI{4}{\joule}}
	{\SI{8}{\joule}}
	{\SI{10.8}{\joule}}
	{\True \SI{2.45}{\joule}}
	\loigiai{}
\end{ex}
% ===================================================================
\begin{ex}
	Xe điện VinFast VF8 được trang bị 2 động cơ điện tại trục trước và trục sau, cho biết công suất tối đa của xe đạt tới \SI{300}{\kilo\watt} ở bản Plus. Giả sử xe đang chạy đều trên đường thẳng với tốc độ \SI{72}{\kilo\meter/\hour}. Lực kéo trung bình của động cơ lúc đó là
	\choice
	{\SI{10800}{\newton}}
	{\SI{144000}{\newton}}
	{\SI{20803}{\newton}}
	{\True \SI{15000}{\newton}}
	\loigiai{}
\end{ex}

% ===================================================================
\begin{ex}
	Một trục kéo có hiệu suất \SI{85}{\percent} được hoạt động bởi một động cơ có công suất \SI{50}{\kilo\watt}. Trục kéo có thể kéo đều một vật có trọng lượng \SI{8000}{\newton} với tốc độ gần bằng
	\choice
	{\SI{10.7}{\meter/\second}}
	{\True \SI{5.3}{\meter/\second}}
	{\SI{6.5}{\meter/\second}}
	{\SI{19.3}{\meter/\second}}
	\loigiai{}
\end{ex}

% ===================================================================
\begin{ex}
	Sau khi cất cánh 0,5 phút, một trực thăng có $m=6$ tấn bay lên đến độ cao $h=\SI{900}{\meter}$. Coi chuyển động của trực thăng là chuyển động thẳng nhanh dần đều. Lấy $g=\SI{10}{\meter/\second^2}$. Công lực kéo do động cơ tạo ra là
	\choice
	{\True \SI{64.8E6}{\joule}}
	{\SI{46.8E6}{\joule}}
	{\SI{86.8E6}{\joule}}
	{\SI{68.8E6}{\joule}}
	\loigiai{}
\end{ex}
% ===================================================================
\begin{ex}
	Một vật \SI{250}{\gram} được đặt trên mặt phẳng nghiêng. Chiều dài của mặt phẳng nghiêng là \SI{10}{\meter}, chiều cao của mặt phẳng nghiêng là \SI{3}{\meter}. Lấy $g=\SI{10}{\meter/\second^2}$. Công của trọng lực khi vật trượt không ma sát từ đỉnh xuống chân mặt phẳng nghiêng là
	\choice
	{\SI{110}{\joule}}
	{\SI{750}{\joule}}
	{\True \SI{7.5}{\joule}}
	{\SI{5.7}{\joule}}
	\loigiai{}
\end{ex}
% ===================================================================
\begin{ex}
	Hòn đá nhỏ được ném thẳng đứng từ mặt đất lên trên với tốc độ $v_0=\SI{2}{\meter/\second}$. Chọn gốc thế năng tại mặt đất. Lấy gia tốc trọng trường $g=\SI{10}{\meter/\second^2}$. Thế năng bằng $\dfrac{1}{4}$ động năng khi vật có độ cao so với mặt đất là
	\choice
	{\SI{20}{\meter}}
	{\SI{5}{\meter}}
	{\SI{4}{\meter}}
	{\True \SI{16}{\meter}}
	\loigiai{}
\end{ex}
% ===================================================================
\begin{ex}
	Một mũi tên khối lượng \SI{100}{\gram} được bắn đi, lực trung bình của dây cung tác dụng vào đuôi mũi tên bằng \SI{90}{\newton} trong suốt khoảng cách \SI{0.8}{\meter}. Mũi tên rời dây cùng với tốc độ gần bằng
	\choice
	{\SI{32}{\meter/\second}}
	{\True \SI{38}{\meter/\second}}
	{\SI{40}{\meter/\second}}
	{\SI{48}{\meter/\second}}
	\loigiai{}
\end{ex}
