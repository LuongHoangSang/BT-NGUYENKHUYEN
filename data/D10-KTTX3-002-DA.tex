\begin{center}
	\begin{tabular}{M{9.25cm}M{8.75cm}}
		\textbf{TRƯỜNG THCS-THPT NGUYỄN KHUYẾN}& \textbf{ÔN TẬP KTTX LẦN 1 - HỌC KÌ 2}\\
		\textbf{MÃ ĐỀ: 002}& \textbf{Bài thi môn: VẬT LÝ 10}\\
		\textit{(Đề thi có 04 trang)}& \textit{Thời gian làm bài: 40 phút, không kể phát đề}
		
		\noindent\rule{4cm}{0.8pt} \\
	\end{tabular}
\end{center}
\setcounter{section}{0}
\begin{center}
	\textbf{\large BẢNG ĐÁP ÁN}
\end{center}
\section{}
\inputansbox{10}{ans/D10-KTTX3-002-TN}
\section{}
\inputansbox[2]{2}{ans/D10-KTTX3-002-TF}
\section{}
\setcounter{ex}{0}
% ======================================================================
\begin{ex}
	\begin{enumerate}[label=\alph*)]
		\item $\SI{36}{\newton}$.
		\item $\SI{48}{\newton}$.
	\end{enumerate}
	\loigiai{}
\end{ex}
% ======================================================================
\begin{ex}
	\begin{enumerate}[label=\alph*)]
		\item $\SI{1568}{\pascal}$.
		\item Mực chất lỏng trong ống chọc tủy sống dâng lên. Huyết áp tăng lên nên áp suất này được truyền đến dịch não tủy, áp suất dịch não tủy tăng nên chiều cao cột chất lỏng trong ống chọc tủy sống tăng $p=\rho gh\rightarrow p\sim h$.
	\end{enumerate}
	\loigiai{}
\end{ex}
