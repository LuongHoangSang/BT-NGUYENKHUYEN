\begin{center}
	\textbf{CHƯƠNG 7: ĐỘNG LƯỢNG VÀ ĐỊNH LUẬT BẢO TOÀN ĐỘNG LƯỢNG}
\end{center}

% ===================================================================
\begin{ex}
	Một vật chuyển động với tốc độ tăng dần thì có	
	\choice
	{động lượng không đổi}
	{động lượng bằng không}
	{\True động lượng tăng dần}
	{động lượng giảm dần}
	\loigiai{}
\end{ex}
% ===================================================================
\begin{ex}
	Chọn phát biểu đúng về mối quan hệ giữa vectơ động lượng $\vec{p}$ và vận tốc $\vec{v}$ của một chất điểm.
	\choice
	{Cùng phương, ngược chiều}
	{\True Cùng phương, cùng chiều}
	{Vuông góc với nhau}
	{Hợp với nhau một góc $\alpha \neq 0$}
	\loigiai{}
\end{ex}




% ===================================================================
\begin{ex}
	Tổng động lượng của hai vật cùng khối lượng chuyển động cùng vận tốc nhưng ngược chiều khi va chạm
	\choice
	{tăng lên}
	{giảm}
	{\True bằng không}
	{là vô hạn}
	\loigiai{}
\end{ex}
% ===================================================================
\begin{ex}
	Trong một vụ va chạm hoàn toàn không đàn hồi, tổng động năng của các vật va chạm
	\choice
	{hoàn toàn biến mất}
	{được tăng lên}
	{\True giảm}
	{không đổi}
	\loigiai{}
\end{ex}
% ===================================================================
\begin{ex}
	Phát biểu nào sau đây \textbf{sai} khi nói về một vật chuyển động tròn đều?
	\choice
	{Quỹ đạo chuyển động là một đường tròn hoặc một phần của đường tròn}
	{Tốc độ của vật không đối theo thời gian}
	{Với tốc độ xác định, bán kính quỹ đạo càng nhỏ thì phương của vận tốc biến đổi càng nhanh}
	{\True Với bán kính quỹ đạo xác định, nếu tốc độ tăng gấp đôi thì gia tốc hướng tâm cũng tăng gấp đôi}
	\loigiai{}
\end{ex}

% ===================================================================
\begin{ex}
	Đơn vị nào sau đây là đơn vị của động lượng?
	\choice
	{\True\si{\newton\cdot\second}}
	{\si{\newton\cdot\meter}}
	{\si{\newton\cdot\meter/\second}}
	{\si{\newton/\second}}
	\loigiai{}
\end{ex}
% ===================================================================
\begin{ex}
	Biểu thức nào sau đây mô tả đúng mối quan hệ giữa động lượng và động năng của vật?
	\choice
	{$p=\sqrt{m\cdot W_{\text{đ}}}$}
	{$p=m\cdot W_{\text{đ}}$}
	{\True $p=\sqrt{2m\cdot W_{\text{đ}}}$}
	{$p=2m\cdot W_{\text{đ}}$}
	\loigiai{}
\end{ex}
% ===================================================================
\begin{ex}
	Trong các quá trình chuyển động sau đây, quá trình nào mà động lượng của vật không thay đổi?
	\choice
	{Vật chuyển động chạm vào vách tường rồi phản xạ lại}
	{Vật được ném ngang}
	{Vật đang rơi tự do}
	{\True Vật chuyển động thẳng đều}
	\loigiai{}
\end{ex}
% ===================================================================
\begin{ex}
	Trong trường hợp nào sau đây, hệ có thể được coi là hệ kín?
	\choice
	{Hai viên bi chuyển động trên mặt phẳng nằm ngang}
	{Hai viên bi chuyển động trên mặt phẳng nghiêng}
	{Hai viên bị rơi thẳng đứng trong không khí}
	{\True Hai viên bi chuyển động không ma sát trên mặt phẳng nằm ngang}
	\loigiai{}
\end{ex}
% ===================================================================
\begin{ex}
	Khi một vật đang rơi (không chịu tác dụng của lực cản không khí) thì
	\choice
	{động lượng của vật không đổi}
	{\True động lượng của vật chỉ thay đổi về độ lớn}
	{động lượng của vật chỉ thay đổi về hướng}
	{động lượng của vật thay đổi cả về hướng và độ lớn}
	\loigiai{}
\end{ex}
% ===================================================================
\begin{ex}
	Hai vật có khối lượng $m_1$ và $m_2$ chuyển động với vận tốc lần lượt là $\vec{v}_1$ và $\vec{v}_2$. Động lượng của hệ là
	\choice
	{$m\cdot\vec{v}$}
	{\True $m_1\cdot\vec{v}_1+m_2\cdot\vec{v}_2$}
	{$0$}
	{$m_1\cdot v_1+m_2\cdot v_2$}
	\loigiai{}
\end{ex}
% ===================================================================
\begin{ex}
	Biểu thức của định luật II Newton có thể viết dưới dạng
	\choice
	{\True $\vec{F}\Delta t=\Delta \vec{p}$}
	{$\vec{F}\Delta p=m\vec{a}$}
	{$\dfrac{\vec{F}\Delta p}{\Delta t}=m\vec{a}$}
	{$\vec{F}\Delta p=\Delta t$}
	\loigiai{}
\end{ex}
% ===================================================================
\begin{ex}
	Chọn từ/cụm từ thích hợp để điền vào chỗ trống trong đoạn dưới đây.\\
	Va chạm mềm (còn gọi là va chạm (1) \dots) xảy ra khi hai vật dính vào nhau và chuyển động với cùng (2) \dots sau va chạm. Động năng của hệ sau va chạm (3) \dots động năng của hệ trước va chạm.
	\choice
	{(1) đàn hồi; (2) vận tốc; (3) bằng}
	{(1) đàn hồi; (2) tốc độ; (3) lớn hơn}
	{\True (1) không đàn hồi; (2) vận tốc; (3) nhỏ hơn}
	{(1) không đàn hồi; (2) tốc độ; (3) bằng}
	\loigiai{}
\end{ex}
% ===================================================================
\begin{ex}
	Va chạm tuyệt đối đàn hồi và va chạm mềm khác nhau ở điểm nào sau đây?
	\choice
	{Hệ va chạm đàn hồi có động lượng bảo toàn còn va chạm mềm thì động lượng không bảo toàn}
	{\True Hệ va chạm đàn hồi có động năng không thay đổi còn va chạm mềm thì động năng thay đổi}
	{Hệ va chạm mềm có động năng không thay đổi còn va chạm đàn hồi thì động năng thay đổi}
	{Hệ va chạm mềm có động lượng bảo toàn còn và chạm đàn hồi thì động lượng không bảo toàn}
	\loigiai{}
\end{ex}
% ===================================================================
\begin{ex}
	Cho hai vật va chạm trực diện với nhau, sau va chạm, hai vật dính liền thành một khối và chuyển động với cùng vận tốc. Động năng của hệ ngay trước va chạm và sau va chạm lần lượt là $W_{\text{đ}}$ và $W^\prime_{\text{đ}}$. Biểu thức nào dưới đây là đúng?
	\choice
	{$W_{\text{đ}}=W^\prime_{\text{đ}}$}
	{$W_{\text{đ}}<W^\prime_{\text{đ}}$}
	{\True $W_{\text{đ}}>W^\prime_{\text{đ}}$}
	{$W_{\text{đ}}=2W^\prime_{\text{đ}}$}
	\loigiai{}
\end{ex}
% ===================================================================
\begin{ex}
	Trong điều kiện nào dưới đây, hai vật chuyển động đến va chạm đàn hồi với nhau và có thể đứng yên sau va chạm?
	\choice
	{Hai vật có cùng khối lượng và chuyển động cùng chiều đến va chạm nhau}
	{Một vật khối lượng rất nhỏ chuyển động đến va chạm với một vật có khối lượng rất lớn đang đứng yên}
	{Hai vật có khối lượng bằng nhau, chuyển động ngược chiều với cùng một tốc độ}
	{\True Không thể xảy ra hiện tượng trên}
	\loigiai{}
\end{ex}
% ===================================================================
\begin{ex}
	Khẳng định nào sau đây \textbf{không đúng} trong trường hợp hai vật cô lập va chạm mềm với nhau?
	\choice
	{Năng lượng của hệ trước và sau va chạm được bảo toàn}
	{\True Cơ năng của hệ trước và sau va chạm được bảo toàn}
	{Động lượng của hệ trước và sau va chạm được bảo toàn}
	{Trong quá trình va chạm, hai vật chịu lực tác dụng như nhau về độ lớn}
	\loigiai{}
\end{ex}
% ===================================================================
\begin{ex}
	Hai vật nhỏ có khối lượng khác nhau ban đầu ở trạng thái nghỉ. Sau đó, hai vật đồng thời chịu tác dụng của ngoại lực không đổi có độ lớn như nhau và bắt đầu chuyển động. Sau cùng một khoảng thời gian, điều nào sau đây là đúng?
	\choice
	{Động năng của hai vật như nhau}
	{Vật có khối lượng lớn hơn có động năng lớn hơn}
	{\True Vật có khối lượng lớn hơn có động năng nhỏ hơn}
	{Không đủ dữ kiện để so sánh}
	\loigiai{}
\end{ex}
% ===================================================================
\begin{ex}
	Vật 1 có khối lượng $m$ đang chuyển động với tốc độ $v_0$ đến va chạm tuyệt đối đàn hồi với vật 2 có cùng khối lượng và đang đứng yên. Nếu khối lượng vật 2 tăng lên gấp đôi thì động năng của hệ sau va chạm
	\choice
	{\True không đổi}
	{tăng 2 lần}
	{giảm 1,5 lần}
	{tăng 1,5 lần}
	\loigiai{}
\end{ex}
% ===================================================================
\begin{ex}
	Thủ môn khi bắt bóng muốn không đau tay và khỏi ngã thì phải co tay lại và lùi người lại một chút theo hướng đi của quả bóng. Thủ môn phải làm thế nhằm mục đích
	\choice
	{giảm thời gian tiếp xúc giữa tay và bóng để làm tăng độ biến thiên động lượng của quả bóng}
	{tăng thời gian tiếp xúc giữa tay và bóng để làm giảm động lượng của quả bóng}
	{giảm thời gian tiếp xúc giữa tay và bóng để làm tăng xung lượng của lực quả bóng tác dụng lên tay}
	{\True tăng thời gian tiếp xúc giữa tay và bóng để làm giảm cường độ của lực quả bóng tác dụng lên tay}
	\loigiai{}
\end{ex}
% ===================================================================
\begin{ex}
	Định luật bảo toàn động lượng
	\choice
	{không tương đương với các định luật Newton}
	{\True tương đương với định luật III Newton}
	{tương đương với định luật I Newton}
	{tương đương với định luật II Newton}
	\loigiai{}
\end{ex}
% ===================================================================
\begin{ex}
	Một vật khối Iượng \SI{500}{\gram} chuyển động thẳng dọc theo trục toạ độ $Ox$ với vận tốc \SI{36}{\kilo\meter/\hour}. Động lượng của vật bằng
	\choice
	{\SI{9}{\kilogram\cdot\meter/\second}}
	{\True \SI{5}{\kilogram\cdot\meter/\second}}
	{\SI{10}{\kilogram\cdot\meter/\second}}
	{\SI{4.5}{\kilogram\cdot\meter/\second}}
	\loigiai{}
\end{ex}
% ===================================================================
\begin{ex}
	Chất điểm khối lượng $m$ chuyển động không vận tốc đầu dưới tác dụng của lực không đổi $\vec{F}$. Động lượng của chất điểm ở thời điểm $t$ là
	\choice
	{$\vec{p}=\vec{F}\cdot m$}
	{\True $\vec{p}=\vec{F}\cdot t$}
	{$\vec{p}=\dfrac{\vec{F}}{m}$}
	{$\vec{p}=\dfrac{\vec{F}}{t}$}
	\loigiai{}
\end{ex}

% ===================================================================
\begin{ex}
	Một chất điểm chuyển động không vận tốc đầu dướí tác dụng của lực không đổi $F=\SI{0.1}{\newton}$. Động lượng của chất điểm ở thời điểm $t=\SI{3}{\second}$ kể từ lúc bắt đầu chuyển động là
	\choice
	{\SI{30}{\kilogram\cdot\meter/\second}}
	{\SI{3}{\kilogram\cdot\meter/\second}}
	{\True \SI{0.3}{\kilogram\cdot\meter/\second}}
	{\SI{0.03}{\kilogram\cdot\meter/\second}}
	\loigiai{}
\end{ex}

% ===================================================================
\begin{ex}
	Một quả bóng khối lượng \SI{250}{\gram} bay tới đập vuông góc vào tường với tốc độ $v_1=\SI{4.5}{\meter/\second}$ và bật ngược trở lại với tốc độ $v_2=\SI{3.5}{\meter/\second}$. Động lượng của vật đã thay đổi một lượng bằng
	\choice
	{\True \SI{2}{\kilogram\cdot\meter/\second}}
	{\SI{5}{\kilogram\cdot\meter/\second}}
	{\SI{1.25}{\kilogram\cdot\meter/\second}}
	{\SI{0.75}{\kilogram\cdot\meter/\second}}
	\loigiai{}
\end{ex}



% ===================================================================
\begin{ex}
	Một đầu đạn khối lượng \SI{10}{\gram} được bắn ra khỏi nòng của một khẩu súng khối lượng\SI{5}{\kilogram} với vận tốc \SI{600}{\meter/\second}. Nếu bỏ qua khối lượng của đầu đạn thì vận tốc giật của súng là
	\choice
	{\SI{1.2}{\centi\meter/\second}}
	{\True\SI{1.2}{\meter/\second}}
	{\SI{12}{\centi\meter/\second}}
	{\SI{12}{\meter/\second}}
	\loigiai{}
\end{ex}

% ===================================================================
\begin{ex}
	Một vật khối lượng \SI{500}{\gram} chuyển động thẳng theo chiều âm trục toạ độ $Ox$ với tốc độ \SI{12}{\meter/\second}. Động lượng của vật có giá trị là
	\choice
	{\SI{6}{\kilogram\cdot\meter/\second}}
	{\SI{-3}{\kilogram\cdot\meter/\second}}
	{\True \SI{-6}{\kilogram\cdot\meter/\second}}
	{\SI{3}{\kilogram\cdot\meter/\second}}
	\loigiai{}
\end{ex}
% ===================================================================
\begin{ex}
	Một chất điểm có khối lượng $m$ bắt đầu trượt không ma sát từ trên mặt phẳng nghiêng xuống. Gọi $\alpha$ là góc của mặt phẳng nghiêng so với mặt phẳng nằm ngang. Động lượng của chất điểm ở thời điểm $t$ là
	\choice
	{\True $p=mgt\sin\alpha$}
	{$p=mgt$}
	{$p=mgt\cos\alpha$}
	{$p=gt\sin\alpha$}
	\loigiai{}
\end{ex}
% ===================================================================
\begin{ex}
	Một vật có khối lượng \SI{1}{\kilogram} trượt không ma sát trên một mặt phẳng ngang với tốc độ =\SI{5}{\meter/\second} đến đập vào một bức tường thẳng đứng theo phương vuông góc với tường. Sau va chạm, vật bật ngược trở lại phương cũ với tốc độ \SI{2}{\meter/\second}. Thời gian tương tác là \SI{0.4}{\second}. Lực $\vec{F}$ do tường tác dụng lên vật có độ lớn bằng
	\choice
	{\SI{1750}{\newton}}
	{\True \SI{17.5}{\newton}}
	{\SI{175}{\newton}}
	{\SI{1.75}{\newton}}
	\loigiai{}
\end{ex}

% ===================================================================
\begin{ex}
	Một viên đạn đang bay với vận tốc \SI{10}{\meter/\second} thì nổ thành hai mảnh. Mảnh thứ nhất, chiếm \SI{60}{\percent} khối lượng của viên đạn và tiếp tục bay theo hướng cũ với vận tốc \SI{25}{\meter/\second}. Tốc độ và hướng chuyển động của mảnh thứ hai là
	\choice
	{\SI{12.5}{\meter/\second}; theo hướng viên đạn ban đầu}
	{\True\SI{12.5}{\meter/\second}; ngược hướng hướng viên đạn ban đầu}
	{\SI{6.25}{\meter/\second}; theo hướng viên đạn ban đầu}
	{\SI{6.25}{\meter/\second}; ngược hướng viên đạn ban đầu}
	\loigiai{}
\end{ex}
