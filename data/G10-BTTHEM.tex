% ======================================================================
\begin{ex}
	Một xe chuyển động thẳng nhanh dần đều đi trên hai đoạn đường liên tiếp bằng nhau $\SI{100}{\meter}$, lần lượt trong $\SI{5}{\second}$ và $\SI{3}{\second}$. Tính gia tốc của xe.
	\loigiai{}
\end{ex}
% ======================================================================
\begin{ex}
	Hai người đi xe đạp khởi hành cùng lúc và đi ngược chiều. Người thứ nhất có vận tốc đầu là $\SI{4.5}{\kilo\meter/\hour}$ và nhanh dần đều với gia tốc $\SI{20}{\centi\meter/\second^2}$. Người thứ hai có vận tốc đầu $\SI{5.4}{\kilo\meter/\hour}$ và đi nhanh dần đều với với gia tốc $\SI{0.2}{\meter/\second^2}$. Khoảng cách ban đầu là $\SI{130}{\meter}$. Xác định thời điểm để hai xe cách nhau $\SI{40}{\meter}$?
	
	\loigiai{}
\end{ex}
% ======================================================================
\begin{ex}
	Một ôtô chuyển động trên đường thẳng, bắt đầu khởi hành nhanh dần đều với gia tốc $a_1=\SI{5}{\meter/\second^2}$, sau đó chuyển động thẳng đều và cuối cùng chuyển động chậm dần đều với gia tốc $a_3 = \SI{5}{\meter/\second^2}$ cho đến khi dừng lại. Thời gian ôtô chuyển động là $\SI{25}{\second}$. Tốc độ trung bình của ô tô trên cả đoạn đường là $\SI{20}{\meter/\second}$. Trong giai đoạn chuyển động thẳng đều ôtô đạt vận tốc là bao nhiêu?
	\loigiai{}
\end{ex}