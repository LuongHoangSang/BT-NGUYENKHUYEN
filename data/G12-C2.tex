\setcounter{section}{0}
\section{Câu trắc nghiệm nhiều phương án lựa chọn}
\textit{Thí sinh trả lời từ câu 1 đến câu 18. Mỗi câu hỏi thí sinh chọn một phương án}
\setcounter{ex}{0}
\Opensolutionfile{ans}[ans/G12C2TN]
% ===================================================================
\begin{ex}
	Hiện tượng nào sau đây \textbf{không} thể hiện rõ thuyết động học phân tử?
	\choice
	{\True Không khí nóng thì nổi lên cao, không khí lạnh chìm xuống trong bầu khí quyển}
	{Mùi nước hoa lan toả trong một căn phòng kín}
	{Chuyển động hỗn loạn của các hạt phấn hoa trong nước yên lặng}
	{Cốc nước được nhỏ mực, sau một thời gian có màu đồng nhất}
	\loigiai{Đáp án A thể hiện sự đối lưu của dòng khí, không thể hiện rõ thuyết động học phân tử.}
\end{ex}
% ===================================================================
\begin{ex}
Đại lượng nào sau đây được giữ không đổi theo định luật Boyle?	
	\choice
	{Chỉ khối lượng khí}
	{Chỉ nhiệt độ khí}
	{Khối lượng khí và áp suất khí}
	{\True Khối lượng khí và nhiệt độ khí}
	\loigiai{}
\end{ex}
% ===================================================================
\begin{ex}
	Công thức liên hệ giữa hằng số Boltzmann $k$ với số Avogadro $N_\text{A}$ và hằng số khí lí tưởng $R$ là 
	\choice
	{$N_\text{A}$}
	{$N_\text{A}R$}
	{\True $R/N_\text{A}$}
	{$N_\text{A}/R$}
	\loigiai{}
\end{ex}
% ===================================================================
\begin{ex}
	Trong hệ toạ độ $\left(p, T\right)$ đường đẳng nhiệt là
	\choice
	{đường thẳng kéo dài qua O}
	{đường cong hyperbol}
	{đường thẳng song song trục $OT$}
	{\True đường thẳng song song trục $Op$}
	\loigiai{}
\end{ex}
% ===================================================================
\begin{ex}
	Khi làm nóng một lượng khí đẳng tích thì	
	\choice
	{áp suất khí không đổi}
	{\True số phân tử trong một đơn vị thể tích không đổi}
	{số phân tử khí trong một đơn vị thể tích tăng tỉ lệ thuận với nhiệt độ}
	{số phân tử khí trong một đơn vị thể tích giảm tỉ lệ nghịch với nhiệt độ.}
	\loigiai{}
\end{ex}
% ===================================================================
\begin{ex}
Một chất khí có thể tích $\SI{5.4}{\liter}$ ở áp suất $\SI{1.06}{atm}$. Giả sử nhiệt độ không thay đổi khi tăng áp suất tới $\SI{1.52}{atm}$ thì khối khí có thể tích bằng bao nhiêu?	
	\choice
	{\True $\SI{3.8}{\liter}$}
{$\SI{5.0}{\liter}$}
	{$\SI{5.4}{\liter}$}
	{$\SI{7.7}{\liter}$}
	\loigiai{ Do nhiệt độ không đổi nên áp dụng định luật Boyle:
	$$p_1V_1=p_2V_2\Rightarrow V_2=\SI{3.8}{\liter}.$$}
\end{ex}
% ===================================================================
\begin{ex}
Trong một quá trình đẳng áp, người ta thực hiện công là $\SI{4.5E4}{\joule}$ làm một lượng khí có thể tích thay đổi từ $\SI{2.6}{\meter^3}$ đến $\SI{1.1}{\meter^3}$. Áp suất khí trong quá trình này là bao nhiêu?	
	\choice
	{$\SI{1.2E4}{\pascal}$}
	{$\SI{2.4E4}{\pascal}$}
	{\True $\SI{3.0E4}{\pascal}$}
	{$\SI{4.1E4}{\pascal}$}
	\loigiai{
Áp suất khí trong quá trình này:
$$p=\dfrac{A}{V_1-V_2}=\SI{3E4}{\pascal}.$$	
}
\end{ex}


% ===================================================================
\begin{ex}
	Hai mol khí lí tưởng ở áp suất $\SI{3.0}{atm}$ và $\SI{10}{\celsius}$ được làm nóng đến $\SI{150}{\celsius}$. Nếu thể tích được giữ không đổi trong quá trình đun nóng này thì áp suất cuối cùng của khí là bao nhiêu?
		\choice
	{\True $\SI{4.5}{atm}$}
	{$\SI{1.8}{atm}$}
	{$\SI{0.14}{atm}$}
	{$\SI{1.0}{atm}$}
	\loigiai{
Vì thể tích khí được giữ không đổi nên:
$$\dfrac{p_1}{T_1}=\dfrac{p_2}{T_2}\Rightarrow p_2=\dfrac{T_2}{T_1}\cdot p_1\approx\SI{4.48}{atm}.$$	
}
\end{ex}
% ===================================================================
\begin{ex}
	Một bình thuỷ tinh kín chịu nhiệt chứa không khí ở điều kiện tiêu chuẩn. Nung nóng bình lên tới $\SI{200}{\celsius}$. Coi sự nở vì nhiệt của bình là không đáng kể. Áp suất không khí trong bình là
	\choice
	{$\SI{7.4E4}{\pascal}$}
	{\True $\SI{17.55E5}{\pascal}$}
	{$\SI{1.28E5}{\pascal}$}
	{$\SI{58467}{\pascal}$}
	\loigiai{
$$\dfrac{p_1}{T_1}=\dfrac{p_2}{T_2}\Rightarrow p_2=\dfrac{T_2}{T_1}p_1=\dfrac{\left(\SI{473}{\kelvin}\right)\cdot\left(\SI{1}{atm}\right)}{\SI{273}{\kelvin}}=\SI{1.73}{atm}\approx\SI{17.5E5}{\pascal}.$$
}
\end{ex}
% ===================================================================
\begin{ex}
	Đun nóng đẳng áp một lượng khí lên đến $\SI{67}{\celsius}$ thì thể tích tăng thêm $1/5$ thể tích ban đầu. Nhiệt độ ban đầu của khí là 
	\choice
	{$\SI{1030}{\celsius}$}
	{$\SI{10.3}{\celsius}$}
	{$\SI{283.3}{\celsius}$}
	{$\SI{103}{\celsius}$}
	\loigiai{
Vì quá trình biến đổi trạng thái là đẳng áp nên:
$$\dfrac{V_1}{T_1}=\dfrac{V_2}{T_2}\Leftrightarrow \dfrac{V_1}{T_1}=\dfrac{V_1+\dfrac{V_1}{5}}{67+273}\Rightarrow T_1=\SI{283.3}{\kelvin}\Rightarrow t_1=\SI{10.3}{\kelvin}.$$	
}
\end{ex}
% ===================================================================
\begin{ex}
	Có $\SI{2.00}{\mole}$ khí nitrogen đựng trong một xilanh kín. Nếu nhiệt độ của khí là $\SI{298}{\kelvin}$, áp suất là $\SI{1.01E6}{\newton/\meter^2}$, thể tích của khí là bao nhiêu? Lấy hằng số khí lí tưởng $R=\SI{8.31}{\joule/\left(\mole\cdot\kelvin\right)}$.
	\choice
	{$\SI{9.80E-3}{\meter^3}$}
	{\True $\SI{4.900E-3}{\meter^3}$}
	{$\SI{17.30E-3}{\meter^3}$}
	{$\SI{8.31E-3}{\meter^3}$}
	\loigiai{Áp dụng phương trình Clapeyron - Mendeleev:
$$pV=nRT\Rightarrow V=\dfrac{nRT}{p}\approx\SI{4.9E-3}{\meter^3}.$$	
}
\end{ex}
% ===================================================================
\begin{ex}
Động năng trung bình của phân tử khí lí tưởng ở $\SI{25}{\celsius}$ có giá trị là
	\choice
	{$\SI{5.2E-22}{\joule}$}
	{\True $\SI{6.2E-21}{\joule}$}
	{$\SI{6.2E23}{\joule}$}
	{$\SI{3.2E23}{\joule}$}
	\loigiai{$W_\text{đ}=\dfrac{3}{2}kT=\SI{6.2E-21}{\joule}.$
	}
\end{ex}
% ===================================================================
\begin{ex}
	Một khối khí ở nhiệt độ $\SI{27}{\celsius}$ có áp suất $p=\SI{3E-9}{\newton/\meter^2}$. Hằng số Boltzmann $k=\SI{1.38E-23}{\joule/\kelvin}$. Số lượng phân tử trên mỗi $\si{\centi\meter^3}$ của khối khí khoảng
	\choice
	{$\SI{E10}{}$}
	{$\SI{E5}{}$}
	{$\SI{E8}{}$}
	{\True $\SI{E11}{}$}
	\loigiai{$\dfrac{N}{V}=\dfrac{p}{kT}\approx\SI{7.25E11}{}$.}
\end{ex}
% ===================================================================
\begin{ex}
	Căn bậc hai của trung bình bình phương tốc độ phân tử của một lượng khí lí tưởng là $v=\sqrt{\overline{v^2}}$. Nếu nhiệt độ của lượng khí tăng gấp đôi thì giá trị này là	
	\choice
	{$v$}
	{$\sqrt{2v}$}
	{$2v$}
	{\True $v\sqrt{2}$}
	\loigiai{Vì $\sqrt{\overline{v^2}}=\sqrt{\dfrac{3RT}{M}}$, nên khi nhiệt độ của lượng khí tăng gấp đôi tì tốc độ căn quân phương của phân tử tăng $\sqrt{2}$ lần.}
\end{ex}
% ===================================================================
\begin{ex}
	Căn bậc hai của trung bình bình phương tốc độ phân tử $\sqrt{\overline{v^2}}$ nitrogen ở $\SI{0}{\celsius}$ là	
	\choice
	{$\SI{243}{\meter/\second}$}
	{$\SI{285}{\meter/\second}$}
	{\True $\SI{493}{\meter/\second}$}
	{$\SI{81}{\meter/\second}$}
	\loigiai{$\sqrt{\overline{v^2}}=\sqrt{\dfrac{3RT}{M}}\approx\SI{493}{\meter/\second}.$}
\end{ex}

% ===================================================================
\begin{ex}
Nếu áp suất của một lượng khí lí tưởng xác định tăng $\SI{2E5}{\pascal}$ thì thể tích biến đổi $\SI{3}{\liter}$. Nếu áp suất của lượng khí trên tăng $\SI{5E5}{\pascal}$ thì thể tích biến đổi $\SI{5}{\liter}$. Biết nhiệt độ không đổi trong các quá trình trên. Áp suất và thể tích ban đầu của khí trên là	
	\choice
	{$\SI{2E5}{\pascal}$; $\SI{8}{\liter}$}
	{\True $\SI{4E5}{\pascal}$; $\SI{9}{\liter}$}
	{$\SI{4E5}{\pascal}$; $\SI{12}{\liter}$}
	{$\SI{2E5}{\pascal}$; $\SI{12}{\liter}$}
	\loigiai{
$\heva{pV=\left(p+\SI{2E5}{}\right)\cdot\left(V-3\right)\\
	pV=\left(p+\SI{5E5}{}\right)\cdot\left(V-5\right)
}	\Rightarrow \heva{
-3p+\SI{2E5}{}V=\SI{5E5}{}\\
-5p+\SI{5E5}{}V=\SI{25E5}{}
}\Rightarrow \heva{
p=\SI{4E5}{\pascal}\\
V=\SI{9}{\liter}
}.$
}
\end{ex}
% ===================================================================
\begin{ex}
Dùng ống bơm để bơm một quả bóng đang bị xẹp, mỗi lần bơm đẩy được $\SI{50}{\centi\meter^3}$ không khí ở áp suất $\SI{1}{atm}$ vào quả bóng. Sau 60 lần bơm quả bóng có dung tích $\SI{2}{\liter}$, coi quá trình bơm nhiệt độ không đổi, áp suất khí trong quả bóng sau khi bơm là	
	\choice
	{$\SI{1.25}{atm}$}
	{\True $\SI{1.5}{atm}$}
	{$\SI{2}{atm}$}
	{$\SI{2.5}{atm}$}
	\loigiai{
		\begin{center}
		\begin{tabular}{C{4cm} C{2cm} C{4cm}}
			\colorbox{yellow}{\textcolor{red}{\textbf{Trạng thái 1}}} & $\xrightarrow[]{T=const}$ & \colorbox{yellow}{\textcolor{red}{\textbf{Trạng thái 2}}}\\
			$p_1=\SI{1}{atm}$ & &$p_2=?$\\
			$V_1=60\cdot\left(\SI{50}{\centi\meter^3}\right)=\SI{3}{\liter}$ & & $V_2=\SI{2}{\liter}$
		\end{tabular}
	\end{center}
$$p_1V_1=p_2V_2\Rightarrow p_2=\SI{1.5}{atm}.$$
}
\end{ex}
% ===================================================================
\begin{ex}
Một lượng không khí bị giam trong ống thuỷ tinh nằm ngang bởi một cột thuỷ ngân có chiều dài $\xsi{h}{\left(\milli\meter Hg\right)}$ như, phần cột khí bị giam trong ống có chiều dài $\ell_0$, $p_0$ là áp suất khí quyển tính theo đơn vị $\si{\milli\meter Hg}$. Dựng ống thẳng đứng, miệng ống hướng lên trên thì chiều dài cột không khí trong ống là
	\choice
	{\True $\dfrac{\ell_0}{1+\dfrac{h}{p_0}}$}
	{$\dfrac{\ell_0}{1-\dfrac{h}{p_0}}$}
	{$\dfrac{\ell_0}{1+\dfrac{h}{2p_0}}$}
	{$\dfrac{\ell_0}{1-\dfrac{h}{2p_0}}$}
	\loigiai{
Áp dụng định luật Boyle:
$$p_0\ell_0=\left(p_0+h\right)\ell\Rightarrow \ell=\dfrac{\ell_0}{1+\dfrac{h}{p_0}}.$$	
}
\end{ex}
\Closesolutionfile{ans}


\section{Câu trắc nghiệm đúng/sai} 
\textit{Thí sinh trả lời từ câu 1 đến câu 4. Trong mỗi ý \textbf{a)}, \textbf{b)}, \textbf{c)}, \textbf{d)} ở mỗi câu, thí sinh chọn đúng hoặc sai}
\setcounter{ex}{0}
\Opensolutionfile{ans}[ans/G12C2TF]
% ===================================================================
\begin{ex}
Trong các phát biểu sau về nội dung thuyết động học phân tử chất khí, phát biểu nào là đúng, phát biểu nào là sai?	
	\choiceTF[t]
	{\True Các phân tử chất khí chuyển động hỗn loạn, không ngừng}
	{Các phân tử chất khí chuyển động xung quanh các vị trí cân bằng cố định}
	{Các phân tử chất khí không va chạm với nhau}
	{\True Các phân tử chất khí gây ra áp suất khi va chạm với thành bình chứa}
	\loigiai{
		
	}
\end{ex}
% ===================================================================
\begin{ex}
Nếu áp dụng định luật Charles cho một khối khí xác định, đại lượng không thay đổi là	
	\choiceTF[t]
	{Nhiệt độ và số mol của khối khí}
	{\True Áp lực lên thành bình}
	{\True Áp suất và số mol của khối khí}
	{Nhiệt độ và thể tích của khối khí}
	\loigiai{
		
	}
\end{ex}
% ===================================================================
\begin{ex}
Khi xây dựng công thức tính áp suất chất khí từ mô hình động học phân tử khí, trong các phát biểu sau đây, phát biểu nào là đúng, phát biểu nào là sai?	
	\choiceTF[t]
	{Trong thời gian giữa hai va chạm liên tiếp với thành bình, động lượng của phân tử khí thay đổi một lượng bằng tích khối lượng phân tử và tốc độ trung bình của nó}
	{\True Giữa hai va chạm với thành bình, phân tử khí chuyển động thẳng đều}
	{Lực gây ra thay đổi động lượng của phân tử khí là lực do phân tử khí tác dụng lên thành bình}
	{\True Các phân tử khí chuyển động không có phương ưu tiên, số phân tử đến va chạm với các mặt của thành bình trong mỗi giây là như nhau}
	\loigiai{
		\begin{itemchoice}
			\itemch Sai, vì động lượng của phân tử khí thay đổi một lượng bằng hai lần tích khối lượng phân tử và tốc độ trung bình của nó.
			\itemch Đúng, do bỏ qua lực tương tác nên giữa hai va chạm với thành bình, phân tử khí chuyển động thẳng đều.
			\itemch Sai, vì theo định luật thứ 2 của Newton, lực gây ra thay đổi động lượng của phan tử khí là lực do thành bình tác dụng lên phân tử khí.
			\itemch Đúng, các phân tử khí chuyển động không có phương ưu tiên, số phân tử đến va chạm với các mặt của thành bình trong mỗi giây là như nhau.
		\end{itemchoice}
	}
\end{ex}
% ===================================================================
\begin{ex}
	Trong các phát biểu sau đây về một lượng khí lí tưởng xác định, phát biểu nào là đúng, phát biểu nào là sai?
	\choiceTF[t]
	{\True Áp suất khí tăng lên bằng cách làm tăng nhiệt độ ở thể tích không đổi, tương ứng động năng trung bình của các phân tử đã tăng theo sự tăng nhiệt độ}
	{\True Khi giữ nhiệt độ không đổi, dù thể tích tăng, áp suất giảm nhưng động năng trung bình của các phân tử vẫn không thay đổi}
	{Khi tốc độ của mỗi phân tử tăng lên gấp đôi, áp suất cũng tăng lên gấp đôi}
	{Khi khối khí giảm nhiệt độ, tương ứng động năng trung bình của các phân tử khí cũng giảm nhưng giảm chậm hơn sự giảm nhiệt độ}
	\loigiai{
		
	}
\end{ex}


\Closesolutionfile{ans}
\section{Câu trắc nghiệm trả lời ngắn} \textit{Thí sinh trả lời từ câu 1 đến câu 6}
\setcounter{ex}{0}
\Opensolutionfile{ans}[ans/G12C2TL]
% ===================================================================
\begin{ex}
	Một phân tử khí lí tưởng đang chuyển động qua tâm một bình cầu có đường kính $d=\SI{0.10}{\meter}$. Trong mỗi giây, phân tử này va chạm vào thành bình cầu $\SI{4000}{}$ lần. Coi rằng phân tử này chỉ va chạm với thành bình và tốc độ của phân tử là không đổi sau mỗi va chạm. Tốc độ chuyển động trung bình của phân tử khí trong bình là bao nhiêu $\si{\meter/\second}$?
	\shortans{400}
	\loigiai{Giữa hai va chạm liên tiếp, phân tử đi được quãng đường $d$. Quãng đường đi được trong 1 giây (sau 4000 va chạm) chính là tốc độ trung bình của phân tử.\\
	Vậy tốc độ trung bình của phân tử là $\overline{v}=\SI{4.0E2}{\meter/\second}$.}
\end{ex}
% ===================================================================
\begin{ex}
	Một lượng khí nitrogen có thể tích giảm từ $\SI{21}{\deci\meter^3}$ xuống $\SI{14}{\deci\meter^3}$ thì áp suất tăng  từ $\SI{80.0}{\kilo\pascal}$ đến $\SI{160.0}{\kilo\pascal}$ và có nhiệt độ là $\SI{300.0}{\kelvin}$. Nhiệt độ ban đầu là bao nhiêu kelvin?
	\shortans{225}
	\loigiai{Áp dụng phương trình trạng thái khí lí tưởng:
	$$\dfrac{p_1V_1}{T_1}=\dfrac{p_2V_2}{T_2}.$$
Thay số: $p_1=\SI{80}{\kilo\pascal}$; $p_2=\SI{160}{\kilo\pascal}$; $V_1=\SI{21}{\deci\meter^3}$; $V_2=\SI{14}{\deci\meter^3}$; $T_2=\SI{300}{\kelvin}$, ta được $T_1=\SI{225}{\kelvin}.$
}
\end{ex}
% ===================================================================
\begin{ex}
Một bình kín có thể tích $\SI{0.10}{\meter^3}$ chứa khí hydrogen ở nhiệt độ $\SI{25}{\celsius}$ và áp suất $\SI{6.0E5}{\pascal}$. Biết khối lượng của phân tử khí hydrogen là $m=\SI{0.33E-26}{\kilogram}$.\\ Một trong các giá trị trung bình đặc trưng cho tốc độ của các phân tử khí thường dùng là căn bậc hai của trung bình bình phương tốc độ phân tử $\sqrt{\overline{v^2}}$. Giá trị này của các phân tử hydrogen trong bình là $\xsi{X\cdot 10^3}{\meter/\second}$. Tìm X \textit{(viết kết quả chỉ gồm hai chữ số)}.
	\shortans{6,2}
	\loigiai{Từ công thức: $pV=NkT$ tính được
$$N=\dfrac{pV}{kT}=\dfrac{\left(\SI{6.0E5}{\pascal}\right)\cdot\left(\SI{0.10}{\meter^3}\right)}{\left(\SI{1.38E-23}{\joule/\kelvin}\right)\cdot\left(25+\SI{273}{\kelvin}\right)}=\SI{1.4E25}.$$
Áp dụng công thức $p=\dfrac{1}{3}\dfrac{Nm\overline{v^2}}{V}$, ta xác định được giá trị trung bình bình phương tốc độ của các phân tử khí hydrogen trong bình là
$$\overline{v^2}=\dfrac{3pV}{Nm}=\dfrac{3\left(\SI{6.0E5}{\pascal}\right)\cdot\left(\SI{0.10}{\meter^3}\right)}{\SI{1.4E25}{}\cdot\left(\SI{0.33E-26}{\kilogram}\right)}=\SI{3.9E7}{\meter^2/\second^2}.$$
Tốc độ căn quân phương:
$$\sqrt{\overline{v^2}}=\SI{6.2E3}{\meter/\second}.$$	
}
\end{ex}

% ===================================================================
\begin{ex}
	\textit{Quả bóng thời tiết} hay còn gọi là \textit{bóng thám không}, là một công cụ quan trọng trong việc thu thập dữ liệu khí tượng phục vụ dự báo thời tiết. Nó hoạt động như sau:
	\begin{itemize}
		\item Thả bóng: Quả bóng được thả từ các địa điểm quan sát trên khắp thế giới, thường là hai lần mỗi ngày vào 0 giờ và 12 giờ quốc tế.
		\item Thu thập dữ liệu: Khi được thả, bóng thám không bắt đầu đo các thông số như nhiệt độ, độ ẩm tương đối, áp suất, tốc độ gió và hướng gió.
		\item Truyền dữ liệu: Các thông tin thu thập được sẽ được truyền về đài quan sát thông qua các thiết bị đo lường và truyền tin gắn trên bóng.
		\item Định vị gió: Bóng thám không có thể đo tốc độ gió bằng radar, sóng vô tuyến, hoặc hệ thống định vị toàn cầu (GPS).
		\item Đạt độ cao lớn: Bóng có thể đạt đến độ cao $\SI{40}{\kilo\meter}$ hoặc hơn, trước khi áp suất giảm dần làm cho quả bóng giãn nở đến giới hạn và vỡ.
		\item Quả bóng thời tiết cung cấp dữ liệu quý giá giúp dự đoán điều kiện thời tiết hiện tại và hỗ trợ các công nghệ dự đoán thời tiết. Đây là một phần không thể thiếu trong hệ thống quan sát toàn cầu về thời tiết.
	\end{itemize}
Quả bóng thời tiết sẽ bị nổ ở áp suất $\SI{27640}{\pascal}$ và thể tích tăng tới $\SI{39.5}{\meter^3}$. Một quả bóng thời tiết được thả vào không gian, khí trong nó có thể tích $\SI{15.8}{\meter^3}$ và áp suất ban đầu bằng $\SI{105000}{\pascal}$ và nhiệt độ $\SI{27}{\celsius}$. Khi quả bóng đó bị nổ, nhiệt độ của khí bằng bao nhiêu $\si{\celsius}$ \textit{(viết kết quả chỉ gồm 2 chữ số)}?
	\shortans{-76}
	\loigiai{
Áp dụng	phương trình trạng thái khí lí tưởng:
$$\dfrac{p_1V_1}{T_1}=\dfrac{p_2V_2}{T_2}$$
với $p_1=\SI{105000}{\pascal}$; $V_1=\SI{15.8}{\meter^3}$; $T_1=\SI{300}{\kelvin}$; $p_2=\SI{27640}{\pascal}$; $V_2=\SI{39.5}{\meter^3}$, ta được:
$T_2\approx\SI{197.43}{\kelvin}\Rightarrow t_2=-\SI{76}{\celsius}.$
}
\end{ex}
% ===================================================================
\begin{ex}
Một bình có thể tích $\SI{22.4E-3}{\meter^3}$ chứa $\SI{1.00}{\mole}$ khí hydrogen ở điều kiện tiêu chuẩn (nhiệt độ là $\SI{0.00}{\celsius}$ và áp suất là $\SI{1.00}{atm}$). Người ta bơm thêm $\SI{1.00}{\mole}$ khí helium cũng ở điều kiện tiêu chuẩn vào bình này. Cho khối lượng riêng ở điều kiện tiêu chuẩn của khí hydrogen và khí helium lần lượt là $\SI{9.00E-2}{\kilogram/\meter^3}$ và $\SI{18.0E-2}{\kilogram/\meter^3}$. Giá trị trung bình của bình phương tốc độ phân tử khí trong bình là bao nhiêu \textit{(viết kết quả theo đơn vị $\SI{E6}{\meter/\second}$ và làm tròn đến 2 chữ số thập phân)}.
	\shortans{2,24}
	\loigiai{
Khối lượng riêng của hỗn hợp khí:
$$\rho=\dfrac{\left(\rho_{\ce{H_2}}n_{\ce{H_2}}+\rho_{\ce{He}}n_{\ce{He}}\right)\cdot\SI{22.4E-3}{} }{\SI{22.4E-3}{}}=\SI{27E-2}{\kilogram/\meter^3}.$$
Theo định lý Dalton, áp suất của hỗn hợp khí:
$$p=p_{\ce{H_2}}+p_{\ce{He}}=\SI{2}{atm}.$$
Giá trị trung bình của bình phương tốc độ phân tử khí trong bình là
$$\overline{v^2}=\dfrac{3p}{\rho}=\dfrac{3\cdot 2\cdot\SI{1.01E5}{\pascal}}{\SI{0.27}{\kilogram/\meter^3}}=\SI{2.24E6}{\meter^2/\second^2}.$$
}
\end{ex}
% ===================================================================
\begin{ex}
	Đại lượng $Nm$ là tổng khối lượng của các phân tử khí, tức là khối lượng của một lượng khí xác định. Ở nhiệt độ phòng, mật độ không khí xấp xỉ $\SI{1.29}{\kilogram/\meter^3}$ ở áp suất $\SI{1.00E5}{\pascal}$. Sử dụng những số liệu này để suy ra giá trị $\sqrt{\overline{v^2}}$ theo đơn vị $\si{\meter/\second}$ \textit{(viết kết quả chỉ gồm ba chữ số)}.
	\shortans{482}
	\loigiai{$p=\dfrac{1}{3}\dfrac{Nm}{V}\overline{v^2}=\dfrac{1}{3}\rho \overline{v^2}\Rightarrow \sqrt{\overline{v^2}}=\sqrt{\dfrac{3p}{\rho}}=\SI{482}{\meter/\second}.$}
\end{ex}
\Closesolutionfile{ans}
\begin{center}
	\textbf{--- HẾT ---}
\end{center}